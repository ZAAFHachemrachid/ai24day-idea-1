\chapter{Error Handling and Recovery}

\section{Overview}
This chapter details the comprehensive error handling and recovery mechanisms implemented throughout the system to ensure robust operation and graceful degradation under various failure conditions.

\section{Critical Error Scenarios}

\subsection{Database Errors}
\begin{lstlisting}[language=Python]
class DatabaseError(Exception):
    """Base class for database errors"""
    pass

class ConnectionError(DatabaseError):
    """Database connection errors"""
    def __init__(self, message, retries=3):
        self.message = message
        self.retries = retries
        super().__init__(self.message)

def handle_database_error(error):
    """Handle database-related errors"""
    if isinstance(error, ConnectionError):
        for attempt in range(error.retries):
            try:
                new_connection = establish_connection()
                return new_connection
            except Exception as e:
                logger.error(f"Retry {attempt + 1} failed: {e}")
                time.sleep(1)
    raise error
\end{lstlisting}

\subsection{Camera Access Errors}
\begin{lstlisting}[language=Python]
class CameraManager:
    def handle_camera_error(self, error):
        """Handle camera access and streaming errors"""
        try:
            # Log error details
            logger.error(f"Camera error: {error}")
            
            # Attempt to reset camera
            self.release_camera()
            time.sleep(1)
            success = self.initialize_camera()
            
            if not success:
                # Fall back to secondary camera
                self.switch_to_backup_camera()
                
        except Exception as e:
            logger.critical(f"Camera recovery failed: {e}")
            raise CameraFatalError("Unable to recover camera")
\end{lstlisting}

\section{Thread Management}

\subsection{Thread Error Recovery}
\begin{lstlisting}[language=Python]
class ThreadManager:
    def handle_thread_error(self, thread_id, error):
        """Handle thread errors and recovery"""
        try:
            # Log error
            logger.error(f"Thread {thread_id} error: {error}")
            
            # Attempt recovery
            if self._can_recover(error):
                self._restart_thread(thread_id)
            else:
                self._shutdown_gracefully()
        except Exception as e:
            logger.critical(f"Recovery failed: {e}")
            raise
            
    def _restart_thread(self, thread_id):
        """Restart a failed thread"""
        with self._lock:
            if thread_id in self.active_threads:
                # Stop the failed thread
                self.active_threads[thread_id].stop()
                
                # Create and start new thread
                new_thread = self._create_thread(thread_id)
                new_thread.start()
                self.active_threads[thread_id] = new_thread
\end{lstlisting}

\section{Memory Management}

\subsection{Memory Error Handling}
\begin{lstlisting}[language=Python]
class MemoryManager:
    def handle_memory_error(self, error):
        """Handle memory-related errors"""
        try:
            # Log memory usage
            current_usage = self._get_memory_usage()
            logger.warning(f"Memory usage: {current_usage}MB")
            
            # Clear caches if memory is high
            if current_usage > self.memory_threshold:
                self._clear_caches()
                gc.collect()
                
            # Reduce buffer sizes if needed
            if current_usage > self.critical_threshold:
                self._reduce_buffer_sizes()
                
        except Exception as e:
            logger.error(f"Memory recovery failed: {e}")
            raise
\end{lstlisting}

\section{Face Detection Errors}

\subsection{Model Error Handling}
\begin{lstlisting}[language=Python]
class FaceDetector:
    def handle_detection_error(self, error):
        """Handle face detection errors"""
        try:
            if isinstance(error, ModelError):
                # Attempt to reload model
                self._reload_model()
            elif isinstance(error, ProcessingError):
                # Adjust processing parameters
                self._adjust_parameters()
            else:
                # Fall back to backup detection method
                self._use_backup_detector()
        except Exception as e:
            logger.error(f"Detection recovery failed: {e}")
            raise
\end{lstlisting}

\section{Error Logging and Monitoring}

\subsection{Logging Implementation}
\begin{lstlisting}[language=Python]
class ErrorLogger:
    def __init__(self):
        self.logger = logging.getLogger(__name__)
        self.setup_logging()
        
    def setup_logging(self):
        """Configure logging settings"""
        formatter = logging.Formatter(
            '%(asctime)s - %(name)s - %(levelname)s - %(message)s'
        )
        
        # File handler
        file_handler = RotatingFileHandler(
            'logs/error.log',
            maxBytes=10*1024*1024,  # 10MB
            backupCount=5
        )
        file_handler.setFormatter(formatter)
        
        # Console handler
        console_handler = logging.StreamHandler()
        console_handler.setFormatter(formatter)
        
        self.logger.addHandler(file_handler)
        self.logger.addHandler(console_handler)
\end{lstlisting}

\section{Recovery Procedures}

\subsection{System Recovery}
\begin{lstlisting}[language=Python]
class SystemRecovery:
    def attempt_recovery(self, error_type, error_data):
        """Coordinate system recovery"""
        recovery_steps = {
            'database': self._recover_database,
            'camera': self._recover_camera,
            'detection': self._recover_detection,
            'memory': self._recover_memory
        }
        
        try:
            if error_type in recovery_steps:
                return recovery_steps[error_type](error_data)
            else:
                logger.error(f"Unknown error type: {error_type}")
                return False
        except Exception as e:
            logger.critical(f"Recovery failed: {e}")
            return False
\end{lstlisting}

\section{Best Practices}

\subsection{Error Prevention}
\begin{enumerate}
    \item Regular system health checks
    \item Proactive resource monitoring
    \item Graceful degradation strategies
    \item Backup systems maintenance
\end{enumerate}

\subsection{Error Recovery Guidelines}
\begin{enumerate}
    \item Log all errors with context
    \item Implement retry mechanisms
    \item Maintain fallback options
    \item Monitor recovery success rates
\end{enumerate}