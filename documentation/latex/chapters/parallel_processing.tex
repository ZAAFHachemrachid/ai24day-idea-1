\chapter{Parallel Processing Architecture}

\section{Overview}
The system employs a sophisticated parallel processing architecture to achieve high performance in real-time face detection and recognition. Multiple thread pools work in concert to handle different aspects of the processing pipeline.

\section{Thread Pools}

\subsection{Face Detection Pool}
\begin{lstlisting}[language=Python]
class FaceDetectionPool:
    """Manages parallel face detection processing"""
    def __init__(self, num_workers: int = 6):
        self.num_workers = num_workers
        self.input_queue = Queue(maxsize=30)
        self.result_queue = Queue()
        self.workers = []
        self.active = True
        self.batch_size = 2
        
    def start(self):
        """Initialize and start worker threads"""
        for _ in range(self.num_workers):
            worker = Thread(target=self._detection_worker)
            worker.daemon = True
            worker.start()
            self.workers.append(worker)
\end{lstlisting}

\subsection{Recognition Pool}
\begin{lstlisting}[language=Python]
class RecognitionPool:
    """Manages parallel face recognition processing"""
    def __init__(self, num_workers: int = 4):
        self.num_workers = num_workers
        self.input_queue = Queue(maxsize=20)
        self.result_queue = Queue()
        self.face_database = FaceDatabase()
        
    def _recognition_worker(self):
        """Worker thread for face recognition"""
        while self.active:
            try:
                face_data = self.input_queue.get(timeout=0.1)
                result = self._process_recognition(face_data)
                self.result_queue.put(result)
            except Empty:
                continue
\end{lstlisting}

\section{Frame Buffer Management}

\subsection{Thread-Safe Frame Buffer}
\begin{lstlisting}[language=Python]
class ThreadSafeFrameBuffer:
    """Thread-safe frame buffer with memory pooling"""
    def __init__(self, max_size: int = 30):
        self.max_size = max_size
        self._buffer = Queue(maxsize=max_size)
        self._frame_pool = []
        self._lock = Lock()
        
    def put_frame(self, frame: np.ndarray) -> bool:
        """Thread-safe frame insertion with memory pooling"""
        with self._lock:
            if self._buffer.full():
                self._handle_full_buffer()
            
            pooled_frame = self._get_pooled_frame(frame)
            return self._buffer.put(pooled_frame, block=False)
\end{lstlisting}

\section{Resource Management}

\subsection{Memory Management}
\begin{lstlisting}[language=Python]
class MemoryManager:
    """Manages memory allocation and pooling"""
    def __init__(self):
        self.frame_pool = []
        self.max_pooled_frames = 50
        self._lock = Lock()
        
    def allocate_frame(self, shape):
        """Allocate frame from pool or create new"""
        with self._lock:
            if self.frame_pool:
                frame = self.frame_pool.pop()
                if frame.shape == shape:
                    return frame
            return np.empty(shape, dtype=np.uint8)
\end{lstlisting}

\section{Thread Synchronization}

\subsection{Thread Coordinator}
\begin{lstlisting}[language=Python]
class ThreadCoordinator:
    """Coordinates thread pool activities"""
    def __init__(self):
        self.pools = []
        self.active = True
        self._lock = Lock()
        self._condition = Condition(self._lock)
        
    def coordinate_processing(self):
        """Coordinate processing across thread pools"""
        with self._lock:
            while self.active:
                pool_status = self._check_pools()
                if self._should_adjust_workers(pool_status):
                    self._rebalance_workers()
                self._condition.wait(timeout=1.0)
\end{lstlisting}

\section{Performance Optimization}

\subsection{Batch Processing}
\begin{lstlisting}[language=Python]
def process_batch(self, items):
    """Process items in batches for better performance"""
    results = []
    batch_size = self.batch_size
    
    # Process in batches
    for i in range(0, len(items), batch_size):
        batch = items[i:i + batch_size]
        batch_results = self._process_batch_items(batch)
        results.extend(batch_results)
    
    return results
\end{lstlisting}

\subsection{Load Balancing}
\begin{lstlisting}[language=Python]
class LoadBalancer:
    """Balances workload across thread pools"""
    def __init__(self):
        self.pools = []
        self.metrics = {}
        
    def monitor_load(self):
        """Monitor and adjust thread pool loads"""
        while True:
            metrics = self._collect_metrics()
            if self._needs_rebalancing(metrics):
                self._rebalance_pools()
            time.sleep(5)
\end{lstlisting}

\section{Performance Metrics}

\subsection{Current Performance}
\begin{itemize}
    \item FPS Range: 10-20 FPS
    \item Detection Pool: 4 workers
    \item Frame Buffer: 10 frames
    \item Display: LANCZOS resampling
\end{itemize}

\subsection{Optimized Targets}
\begin{itemize}
    \item Target FPS: 45 FPS average
    \item Enhanced Worker Pool: 6 workers
    \item Increased Buffer Size: 30 frames
    \item Processing Optimizations: +15-20 FPS
\end{itemize}

\section{Error Handling}

\subsection{Thread Error Recovery}
\begin{lstlisting}[language=Python]
def handle_thread_error(self, thread_id, error):
    """Handle thread errors and recovery"""
    try:
        logger.error(f"Thread {thread_id} error: {error}")
        if self._can_recover(error):
            self._restart_thread(thread_id)
        else:
            self._shutdown_gracefully()
    except Exception as e:
        logger.critical(f"Recovery failed: {e}")
        raise
\end{lstlisting}