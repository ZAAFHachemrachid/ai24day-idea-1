\chapter{Performance Optimization}

\section{Overview}
This chapter details the performance optimization strategies implemented across different components of the system to achieve optimal processing speed and resource utilization.

\section{Current Performance Metrics}

\subsection{Baseline Performance}
\begin{itemize}
    \item FPS Range: 10-20 FPS
    \item Detection Pool: 4 worker threads
    \item Frame Buffer Size: 10 frames
    \item Display: LANCZOS resampling
\end{itemize}

\subsection{Target Performance}
\begin{itemize}
    \item Target FPS: 45 FPS average
    \item Enhanced Worker Pool: 6 workers
    \item Increased Buffer Size: 30 frames
    \item Display: BILINEAR resampling
\end{itemize}

\section{Processing Optimizations}

\subsection{Worker Thread Enhancement}
\begin{lstlisting}[language=Python]
# detection_pool.py changes
class FaceDetectionPool:
    def __init__(self, num_workers: int = 6):  # Increased from 4
        self.num_workers = num_workers
        self.skip_frames = False
        self.batch_size = 2
        
    def process_frame(self, frame: np.ndarray) -> int:
        if self.skip_frames and self.input_queue.qsize() > self.num_workers * 2:
            return -1  # Skip frame if we're falling behind
        # ... rest of implementation
\end{lstlisting}

\subsection{Batch Processing}
\begin{lstlisting}[language=Python]
def _detection_worker(self):
    while self.active:
        frames = []
        frame_ids = []
        
        # Collect batch of frames
        for _ in range(self.batch_size):
            try:
                frame_id, frame = self.input_queue.get(timeout=0.1)
                frames.append(frame)
                frame_ids.append(frame_id)
            except:
                break
                
        if frames:
            # Process batch
            results = face_app.get_batch(frames)
            # Queue results
            for i, faces in enumerate(results):
                result = DetectionResult(faces, frame_ids[i], processing_time)
                self.result_queue.put(result)
\end{lstlisting}

\section{Memory Management}

\subsection{Enhanced Frame Buffer}
\begin{lstlisting}[language=Python]
class ThreadSafeFrameBuffer:
    def __init__(self, max_size: int = 30):  # Increased from 10
        self.max_size = max_size
        self._frame_pool = []  # Memory pool for frame objects
        
    def _get_frame_from_pool(self):
        if self._frame_pool:
            return self._frame_pool.pop()
        return None
        
    def _return_frame_to_pool(self, frame):
        if len(self._frame_pool) < self.max_size:
            self._frame_pool.append(frame)
\end{lstlisting}

\section{Display Pipeline Optimization}

\subsection{Optimized Image Processing}
\begin{lstlisting}[language=Python]
class CameraFrame:
    def __init__(self):
        self._last_dimensions = None
        self._cached_image = None
        
    def update_frame(self, frame):
        # Convert color space more efficiently
        if frame.shape[2] == 3:
            frame = cv2.cvtColor(frame, cv2.COLOR_BGR2RGB)
            
        # Cache resized images
        current_dims = (self.winfo_width(), self.winfo_height())
        if self._last_dimensions != current_dims:
            self._cached_image = None
            
        if self._cached_image is None:
            # Use BILINEAR resampling instead of LANCZOS
            image = cv2.resize(frame, current_dims, 
                           interpolation=cv2.INTER_LINEAR)
            self._cached_image = ImageTk.PhotoImage(
                Image.fromarray(image)
            )
            self._last_dimensions = current_dims
\end{lstlisting}

\section{Tracking Optimization}

\subsection{Motion Prediction}
\begin{lstlisting}[language=Python]
class MotionPredictor:
    def predict_next_position(self):
        prediction = self.kalman.predict()
        
        if self.last_position is not None:
            error = self._calculate_prediction_error(
                prediction,
                self.last_position
            )
            self.confidence *= (1 - min(error, 0.5))
            
        return prediction, self.confidence
\end{lstlisting}

\section{Database Optimization}

\subsection{Query Optimization}
\begin{lstlisting}[language=SQL]
-- Optimized user lookup with index
CREATE INDEX idx_users_name ON users(name);
SELECT id, name FROM users WHERE name LIKE ? LIMIT 1;

-- Efficient attendance query
CREATE INDEX idx_attendance_user_date 
ON attendance(user_id, date);
SELECT * FROM attendance 
WHERE user_id = ? AND date = CURRENT_DATE;
\end{lstlisting}

\section{Resource Management}

\subsection{Memory Pooling}
\begin{lstlisting}[language=Python]
class MemoryManager:
    def __init__(self):
        self.frame_pool = []
        self.max_pooled_frames = 50
        self._lock = Lock()
        
    def allocate_frame(self, shape):
        with self._lock:
            if self.frame_pool:
                frame = self.frame_pool.pop()
                if frame.shape == shape:
                    return frame
            return np.empty(shape, dtype=np.uint8)
\end{lstlisting}

\section{Performance Monitoring}

\subsection{Metrics Collection}
\begin{lstlisting}[language=Python]
class PerformanceMonitor:
    def collect_metrics(self):
        metrics = {
            'fps': self._calculate_fps(),
            'processing_time': self._get_processing_time(),
            'memory_usage': self._get_memory_usage(),
            'thread_utilization': self._get_thread_stats()
        }
        return metrics
\end{lstlisting}

\section{Implementation Steps}

\subsection{Phase 1 - Processing Optimizations}
\begin{enumerate}
    \item Update DetectionPool initialization
    \item Implement frame skipping
    \item Add batch processing support
\end{enumerate}

\subsection{Phase 2 - Memory Management}
\begin{enumerate}
    \item Increase buffer size
    \item Implement smart frame dropping
    \item Add memory pooling
\end{enumerate}

\subsection{Phase 3 - Display Pipeline}
\begin{enumerate}
    \item Update resampling method
    \item Add image caching
    \item Optimize color conversion
\end{enumerate}

\section{Testing and Validation}

\subsection{Performance Testing}
\begin{enumerate}
    \item Measure FPS using cv2.getTickCount()
    \item Monitor CPU and memory usage
    \item Verify face detection accuracy
    \item Check for frame dropping or stuttering
\end{enumerate}

\subsection{Expected Improvements}
\begin{itemize}
    \item Processing Optimizations: +15-20 FPS
    \item Memory Management: +5-8 FPS
    \item Display Pipeline: +8-10 FPS
\end{itemize}