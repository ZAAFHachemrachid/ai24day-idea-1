\chapter{Face Tracking System}

\section{Overview}
The tracking system optimizes performance by predicting face locations between frames, reducing the need for full detection on every frame.

\begin{figure}[H]
\centering
\begin{tikzpicture}[node distance=2cm]
    \node (input) [draw, rectangle] {Frame Input};
    \node (predict) [draw, rectangle, right of=input] {Motion Prediction};
    \node (check) [draw, diamond, right of=predict] {Confidence Check};
    \node (track) [draw, rectangle, below of=check] {Update Tracking};
    \node (detect) [draw, rectangle, right of=check] {Full Detection};
    \node (next) [draw, rectangle, right of=track] {Next Frame};
    
    \draw[->] (input) -- (predict);
    \draw[->] (predict) -- (check);
    \draw[->] (check) -- node[right] {Low} (detect);
    \draw[->] (check) -- node[left] {High} (track);
    \draw[->] (detect) -- (next);
    \draw[->] (track) -- (next);
\end{tikzpicture}
\caption{Face Tracking Pipeline}
\end{figure}

\section{Motion Predictor}

\subsection{Kalman Filter Implementation}
\begin{lstlisting}[language=Python]
class MotionPredictor:
    """Predicts face motion between frames using Kalman filtering"""
    def __init__(self):
        self.kalman = cv2.KalmanFilter(4, 2)  # State: x, y, dx, dy
        self.initialize_kalman()
        self.last_position = None
        self.confidence = 1.0
        
    def initialize_kalman(self):
        """Initialize Kalman filter parameters"""
        self.kalman.measurementMatrix = np.array([
            [1, 0, 0, 0],
            [0, 1, 0, 0]], np.float32)
        
        self.kalman.transitionMatrix = np.array([
            [1, 0, 1, 0],
            [0, 1, 0, 1],
            [0, 0, 1, 0],
            [0, 0, 0, 1]], np.float32)
        
        self.kalman.processNoiseCov = np.array([
            [1e-4, 0, 0, 0],
            [0, 1e-4, 0, 0],
            [0, 0, 1e-4, 0],
            [0, 0, 0, 1e-4]], np.float32) * 0.03
\end{lstlisting}

\section{Face Tracker}

\subsection{Tracker Implementation}
\begin{lstlisting}[language=Python]
class FaceTracker:
    """Manages face tracking across video frames"""
    def __init__(self):
        self.trackers = {}  # id -> tracker mapping
        self.motion_predictors = {}  # id -> predictor mapping
        self.max_tracking_age = 30  # frames
        
    def update(self, frame, detections):
        """Update trackers with new frame and detections"""
        # Update existing trackers
        self._update_trackers(frame)
        
        # Associate detections with trackers
        self._associate_detections(detections)
        
        # Update motion predictions
        self._update_predictions()
        
        # Remove stale trackers
        self._cleanup_trackers()
\end{lstlisting}

\subsection{Detection Association}
\begin{lstlisting}[language=Python]
def _associate_detections(self, detections):
    """Associate new detections with existing trackers"""
    if not detections:
        return
        
    # Calculate IoU matrix
    iou_matrix = np.zeros((len(self.trackers), len(detections)))
    for t, tracker in enumerate(self.trackers.values()):
        for d, detection in enumerate(detections):
            iou_matrix[t, d] = self._calculate_iou(
                tracker.get_position(),
                detection
            )
            
    # Use Hungarian algorithm for optimal assignment
    matched_indices = linear_assignment(-iou_matrix)
    
    # Update trackers with matched detections
    for tracker_idx, detection_idx in matched_indices:
        if iou_matrix[tracker_idx, detection_idx] >= self.iou_threshold:
            self._update_tracker_with_detection(
                list(self.trackers.keys())[tracker_idx],
                detections[detection_idx]
            )
\end{lstlisting}

\section{Multi-Scale Tracking}

\subsection{Scale-Adaptive Tracking}
\begin{lstlisting}[language=Python]
class MultiScaleTracker:
    """Implements multi-scale tracking for improved accuracy"""
    def __init__(self):
        self.scales = [0.5, 1.0, 2.0]
        self.current_scale = 1.0
        
    def update(self, frame, bbox):
        """Update tracking at multiple scales"""
        best_response = float('-inf')
        best_bbox = None
        
        for scale in self.scales:
            scaled_frame = cv2.resize(
                frame, 
                None, 
                fx=scale, 
                fy=scale
            )
            scaled_bbox = self._scale_bbox(bbox, scale)
            
            response = self._compute_tracking_response(
                scaled_frame,
                scaled_bbox
            )
            
            if response > best_response:
                best_response = response
                best_bbox = self._scale_bbox(
                    scaled_bbox,
                    1/scale
                )
                
        return best_bbox
\end{lstlisting}

\section{Performance Optimization}

\subsection{Selective Detection}
\begin{lstlisting}[language=Python]
def should_run_detection(self, tracker_id):
    """Determine if full detection is needed"""
    if tracker_id not in self.motion_predictors:
        return True
        
    predictor = self.motion_predictors[tracker_id]
    
    # Run detection if confidence is low
    if predictor.confidence < 0.6:
        return True
        
    # Run detection periodically even with high confidence
    if self.frame_count % 30 == 0:
        return True
        
    return False
\end{lstlisting}

\subsection{Performance Metrics}
\begin{lstlisting}[language=Python]
class TrackingMetrics:
    """Collects and analyzes tracking performance metrics"""
    def __init__(self):
        self.metrics = {
            'tracking_success_rate': [],
            'prediction_accuracy': [],
            'detection_frequency': [],
            'tracking_time': []
        }
        
    def update_metrics(self, tracking_result):
        """Update tracking performance metrics"""
        self.metrics['tracking_success_rate'].append(
            tracking_result.success_rate
        )
        self.metrics['prediction_accuracy'].append(
            tracking_result.prediction_accuracy
        )
        self.metrics['detection_frequency'].append(
            tracking_result.detection_frequency
        )
        self.metrics['tracking_time'].append(
            tracking_result.processing_time
        )
\end{lstlisting}

\section{Best Practices}

\subsection{Tracking Guidelines}
\begin{enumerate}
    \item \textbf{Initialization}
    \begin{itemize}
        \item Initialize trackers only with high-confidence detections
        \item Verify tracking stability before relying on predictions
        \item Maintain multiple tracking hypotheses initially
    \end{itemize}
    
    \item \textbf{Motion Prediction}
    \begin{itemize}
        \item Use Kalman filtering for smooth predictions
        \item Adapt prediction parameters to motion patterns
        \item Maintain prediction confidence metrics
    \end{itemize}
    
    \item \textbf{Error Recovery}
    \begin{itemize}
        \item Implement robust failure detection
        \item Recover gracefully from tracking losses
        \item Maintain backup tracking strategies
    \end{itemize}
\end{enumerate}